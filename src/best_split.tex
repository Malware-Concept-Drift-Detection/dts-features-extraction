\documentclass[11pt]{article}

    \usepackage[breakable]{tcolorbox}
    \usepackage{parskip} % Stop auto-indenting (to mimic markdown behaviour)
    

    % Basic figure setup, for now with no caption control since it's done
    % automatically by Pandoc (which extracts ![](path) syntax from Markdown).
    \usepackage{graphicx}
    % Keep aspect ratio if custom image width or height is specified
    \setkeys{Gin}{keepaspectratio}
    % Maintain compatibility with old templates. Remove in nbconvert 6.0
    \let\Oldincludegraphics\includegraphics
    % Ensure that by default, figures have no caption (until we provide a
    % proper Figure object with a Caption API and a way to capture that
    % in the conversion process - todo).
    \usepackage{caption}
    \DeclareCaptionFormat{nocaption}{}
    \captionsetup{format=nocaption,aboveskip=0pt,belowskip=0pt}

    \usepackage{float}
    \floatplacement{figure}{H} % forces figures to be placed at the correct location
    \usepackage{xcolor} % Allow colors to be defined
    \usepackage{enumerate} % Needed for markdown enumerations to work
    \usepackage{geometry} % Used to adjust the document margins
    \usepackage{amsmath} % Equations
    \usepackage{amssymb} % Equations
    \usepackage{textcomp} % defines textquotesingle
    % Hack from http://tex.stackexchange.com/a/47451/13684:
    \AtBeginDocument{%
        \def\PYZsq{\textquotesingle}% Upright quotes in Pygmentized code
    }
    \usepackage{upquote} % Upright quotes for verbatim code
    \usepackage{eurosym} % defines \euro

    \usepackage{iftex}
    \ifPDFTeX
        \usepackage[T1]{fontenc}
        \IfFileExists{alphabeta.sty}{
              \usepackage{alphabeta}
          }{
              \usepackage[mathletters]{ucs}
              \usepackage[utf8x]{inputenc}
          }
    \else
        \usepackage{fontspec}
        \usepackage{unicode-math}
    \fi

    \usepackage{fancyvrb} % verbatim replacement that allows latex
    \usepackage{grffile} % extends the file name processing of package graphics
                         % to support a larger range
    \makeatletter % fix for old versions of grffile with XeLaTeX
    \@ifpackagelater{grffile}{2019/11/01}
    {
      % Do nothing on new versions
    }
    {
      \def\Gread@@xetex#1{%
        \IfFileExists{"\Gin@base".bb}%
        {\Gread@eps{\Gin@base.bb}}%
        {\Gread@@xetex@aux#1}%
      }
    }
    \makeatother
    \usepackage[Export]{adjustbox} % Used to constrain images to a maximum size
    \adjustboxset{max size={0.9\linewidth}{0.9\paperheight}}

    % The hyperref package gives us a pdf with properly built
    % internal navigation ('pdf bookmarks' for the table of contents,
    % internal cross-reference links, web links for URLs, etc.)
    \usepackage{hyperref}
    % The default LaTeX title has an obnoxious amount of whitespace. By default,
    % titling removes some of it. It also provides customization options.
    \usepackage{titling}
    \usepackage{longtable} % longtable support required by pandoc >1.10
    \usepackage{booktabs}  % table support for pandoc > 1.12.2
    \usepackage{array}     % table support for pandoc >= 2.11.3
    \usepackage{calc}      % table minipage width calculation for pandoc >= 2.11.1
    \usepackage[inline]{enumitem} % IRkernel/repr support (it uses the enumerate* environment)
    \usepackage[normalem]{ulem} % ulem is needed to support strikethroughs (\sout)
                                % normalem makes italics be italics, not underlines
    \usepackage{soul}      % strikethrough (\st) support for pandoc >= 3.0.0
    \usepackage{mathrsfs}
    

    
    % Colors for the hyperref package
    \definecolor{urlcolor}{rgb}{0,.145,.698}
    \definecolor{linkcolor}{rgb}{.71,0.21,0.01}
    \definecolor{citecolor}{rgb}{.12,.54,.11}

    % ANSI colors
    \definecolor{ansi-black}{HTML}{3E424D}
    \definecolor{ansi-black-intense}{HTML}{282C36}
    \definecolor{ansi-red}{HTML}{E75C58}
    \definecolor{ansi-red-intense}{HTML}{B22B31}
    \definecolor{ansi-green}{HTML}{00A250}
    \definecolor{ansi-green-intense}{HTML}{007427}
    \definecolor{ansi-yellow}{HTML}{DDB62B}
    \definecolor{ansi-yellow-intense}{HTML}{B27D12}
    \definecolor{ansi-blue}{HTML}{208FFB}
    \definecolor{ansi-blue-intense}{HTML}{0065CA}
    \definecolor{ansi-magenta}{HTML}{D160C4}
    \definecolor{ansi-magenta-intense}{HTML}{A03196}
    \definecolor{ansi-cyan}{HTML}{60C6C8}
    \definecolor{ansi-cyan-intense}{HTML}{258F8F}
    \definecolor{ansi-white}{HTML}{C5C1B4}
    \definecolor{ansi-white-intense}{HTML}{A1A6B2}
    \definecolor{ansi-default-inverse-fg}{HTML}{FFFFFF}
    \definecolor{ansi-default-inverse-bg}{HTML}{000000}

    % common color for the border for error outputs.
    \definecolor{outerrorbackground}{HTML}{FFDFDF}

    % commands and environments needed by pandoc snippets
    % extracted from the output of `pandoc -s`
    \providecommand{\tightlist}{%
      \setlength{\itemsep}{0pt}\setlength{\parskip}{0pt}}
    \DefineVerbatimEnvironment{Highlighting}{Verbatim}{commandchars=\\\{\}}
    % Add ',fontsize=\small' for more characters per line
    \newenvironment{Shaded}{}{}
    \newcommand{\KeywordTok}[1]{\textcolor[rgb]{0.00,0.44,0.13}{\textbf{{#1}}}}
    \newcommand{\DataTypeTok}[1]{\textcolor[rgb]{0.56,0.13,0.00}{{#1}}}
    \newcommand{\DecValTok}[1]{\textcolor[rgb]{0.25,0.63,0.44}{{#1}}}
    \newcommand{\BaseNTok}[1]{\textcolor[rgb]{0.25,0.63,0.44}{{#1}}}
    \newcommand{\FloatTok}[1]{\textcolor[rgb]{0.25,0.63,0.44}{{#1}}}
    \newcommand{\CharTok}[1]{\textcolor[rgb]{0.25,0.44,0.63}{{#1}}}
    \newcommand{\StringTok}[1]{\textcolor[rgb]{0.25,0.44,0.63}{{#1}}}
    \newcommand{\CommentTok}[1]{\textcolor[rgb]{0.38,0.63,0.69}{\textit{{#1}}}}
    \newcommand{\OtherTok}[1]{\textcolor[rgb]{0.00,0.44,0.13}{{#1}}}
    \newcommand{\AlertTok}[1]{\textcolor[rgb]{1.00,0.00,0.00}{\textbf{{#1}}}}
    \newcommand{\FunctionTok}[1]{\textcolor[rgb]{0.02,0.16,0.49}{{#1}}}
    \newcommand{\RegionMarkerTok}[1]{{#1}}
    \newcommand{\ErrorTok}[1]{\textcolor[rgb]{1.00,0.00,0.00}{\textbf{{#1}}}}
    \newcommand{\NormalTok}[1]{{#1}}

    % Additional commands for more recent versions of Pandoc
    \newcommand{\ConstantTok}[1]{\textcolor[rgb]{0.53,0.00,0.00}{{#1}}}
    \newcommand{\SpecialCharTok}[1]{\textcolor[rgb]{0.25,0.44,0.63}{{#1}}}
    \newcommand{\VerbatimStringTok}[1]{\textcolor[rgb]{0.25,0.44,0.63}{{#1}}}
    \newcommand{\SpecialStringTok}[1]{\textcolor[rgb]{0.73,0.40,0.53}{{#1}}}
    \newcommand{\ImportTok}[1]{{#1}}
    \newcommand{\DocumentationTok}[1]{\textcolor[rgb]{0.73,0.13,0.13}{\textit{{#1}}}}
    \newcommand{\AnnotationTok}[1]{\textcolor[rgb]{0.38,0.63,0.69}{\textbf{\textit{{#1}}}}}
    \newcommand{\CommentVarTok}[1]{\textcolor[rgb]{0.38,0.63,0.69}{\textbf{\textit{{#1}}}}}
    \newcommand{\VariableTok}[1]{\textcolor[rgb]{0.10,0.09,0.49}{{#1}}}
    \newcommand{\ControlFlowTok}[1]{\textcolor[rgb]{0.00,0.44,0.13}{\textbf{{#1}}}}
    \newcommand{\OperatorTok}[1]{\textcolor[rgb]{0.40,0.40,0.40}{{#1}}}
    \newcommand{\BuiltInTok}[1]{{#1}}
    \newcommand{\ExtensionTok}[1]{{#1}}
    \newcommand{\PreprocessorTok}[1]{\textcolor[rgb]{0.74,0.48,0.00}{{#1}}}
    \newcommand{\AttributeTok}[1]{\textcolor[rgb]{0.49,0.56,0.16}{{#1}}}
    \newcommand{\InformationTok}[1]{\textcolor[rgb]{0.38,0.63,0.69}{\textbf{\textit{{#1}}}}}
    \newcommand{\WarningTok}[1]{\textcolor[rgb]{0.38,0.63,0.69}{\textbf{\textit{{#1}}}}}


    % Define a nice break command that doesn't care if a line doesn't already
    % exist.
    \def\br{\hspace*{\fill} \\* }
    % Math Jax compatibility definitions
    \def\gt{>}
    \def\lt{<}
    \let\Oldtex\TeX
    \let\Oldlatex\LaTeX
    \renewcommand{\TeX}{\textrm{\Oldtex}}
    \renewcommand{\LaTeX}{\textrm{\Oldlatex}}
    % Document parameters
    % Document title
    \title{best\_split}
    
    
    
    
    
    
    
% Pygments definitions
\makeatletter
\def\PY@reset{\let\PY@it=\relax \let\PY@bf=\relax%
    \let\PY@ul=\relax \let\PY@tc=\relax%
    \let\PY@bc=\relax \let\PY@ff=\relax}
\def\PY@tok#1{\csname PY@tok@#1\endcsname}
\def\PY@toks#1+{\ifx\relax#1\empty\else%
    \PY@tok{#1}\expandafter\PY@toks\fi}
\def\PY@do#1{\PY@bc{\PY@tc{\PY@ul{%
    \PY@it{\PY@bf{\PY@ff{#1}}}}}}}
\def\PY#1#2{\PY@reset\PY@toks#1+\relax+\PY@do{#2}}

\@namedef{PY@tok@w}{\def\PY@tc##1{\textcolor[rgb]{0.73,0.73,0.73}{##1}}}
\@namedef{PY@tok@c}{\let\PY@it=\textit\def\PY@tc##1{\textcolor[rgb]{0.24,0.48,0.48}{##1}}}
\@namedef{PY@tok@cp}{\def\PY@tc##1{\textcolor[rgb]{0.61,0.40,0.00}{##1}}}
\@namedef{PY@tok@k}{\let\PY@bf=\textbf\def\PY@tc##1{\textcolor[rgb]{0.00,0.50,0.00}{##1}}}
\@namedef{PY@tok@kp}{\def\PY@tc##1{\textcolor[rgb]{0.00,0.50,0.00}{##1}}}
\@namedef{PY@tok@kt}{\def\PY@tc##1{\textcolor[rgb]{0.69,0.00,0.25}{##1}}}
\@namedef{PY@tok@o}{\def\PY@tc##1{\textcolor[rgb]{0.40,0.40,0.40}{##1}}}
\@namedef{PY@tok@ow}{\let\PY@bf=\textbf\def\PY@tc##1{\textcolor[rgb]{0.67,0.13,1.00}{##1}}}
\@namedef{PY@tok@nb}{\def\PY@tc##1{\textcolor[rgb]{0.00,0.50,0.00}{##1}}}
\@namedef{PY@tok@nf}{\def\PY@tc##1{\textcolor[rgb]{0.00,0.00,1.00}{##1}}}
\@namedef{PY@tok@nc}{\let\PY@bf=\textbf\def\PY@tc##1{\textcolor[rgb]{0.00,0.00,1.00}{##1}}}
\@namedef{PY@tok@nn}{\let\PY@bf=\textbf\def\PY@tc##1{\textcolor[rgb]{0.00,0.00,1.00}{##1}}}
\@namedef{PY@tok@ne}{\let\PY@bf=\textbf\def\PY@tc##1{\textcolor[rgb]{0.80,0.25,0.22}{##1}}}
\@namedef{PY@tok@nv}{\def\PY@tc##1{\textcolor[rgb]{0.10,0.09,0.49}{##1}}}
\@namedef{PY@tok@no}{\def\PY@tc##1{\textcolor[rgb]{0.53,0.00,0.00}{##1}}}
\@namedef{PY@tok@nl}{\def\PY@tc##1{\textcolor[rgb]{0.46,0.46,0.00}{##1}}}
\@namedef{PY@tok@ni}{\let\PY@bf=\textbf\def\PY@tc##1{\textcolor[rgb]{0.44,0.44,0.44}{##1}}}
\@namedef{PY@tok@na}{\def\PY@tc##1{\textcolor[rgb]{0.41,0.47,0.13}{##1}}}
\@namedef{PY@tok@nt}{\let\PY@bf=\textbf\def\PY@tc##1{\textcolor[rgb]{0.00,0.50,0.00}{##1}}}
\@namedef{PY@tok@nd}{\def\PY@tc##1{\textcolor[rgb]{0.67,0.13,1.00}{##1}}}
\@namedef{PY@tok@s}{\def\PY@tc##1{\textcolor[rgb]{0.73,0.13,0.13}{##1}}}
\@namedef{PY@tok@sd}{\let\PY@it=\textit\def\PY@tc##1{\textcolor[rgb]{0.73,0.13,0.13}{##1}}}
\@namedef{PY@tok@si}{\let\PY@bf=\textbf\def\PY@tc##1{\textcolor[rgb]{0.64,0.35,0.47}{##1}}}
\@namedef{PY@tok@se}{\let\PY@bf=\textbf\def\PY@tc##1{\textcolor[rgb]{0.67,0.36,0.12}{##1}}}
\@namedef{PY@tok@sr}{\def\PY@tc##1{\textcolor[rgb]{0.64,0.35,0.47}{##1}}}
\@namedef{PY@tok@ss}{\def\PY@tc##1{\textcolor[rgb]{0.10,0.09,0.49}{##1}}}
\@namedef{PY@tok@sx}{\def\PY@tc##1{\textcolor[rgb]{0.00,0.50,0.00}{##1}}}
\@namedef{PY@tok@m}{\def\PY@tc##1{\textcolor[rgb]{0.40,0.40,0.40}{##1}}}
\@namedef{PY@tok@gh}{\let\PY@bf=\textbf\def\PY@tc##1{\textcolor[rgb]{0.00,0.00,0.50}{##1}}}
\@namedef{PY@tok@gu}{\let\PY@bf=\textbf\def\PY@tc##1{\textcolor[rgb]{0.50,0.00,0.50}{##1}}}
\@namedef{PY@tok@gd}{\def\PY@tc##1{\textcolor[rgb]{0.63,0.00,0.00}{##1}}}
\@namedef{PY@tok@gi}{\def\PY@tc##1{\textcolor[rgb]{0.00,0.52,0.00}{##1}}}
\@namedef{PY@tok@gr}{\def\PY@tc##1{\textcolor[rgb]{0.89,0.00,0.00}{##1}}}
\@namedef{PY@tok@ge}{\let\PY@it=\textit}
\@namedef{PY@tok@gs}{\let\PY@bf=\textbf}
\@namedef{PY@tok@ges}{\let\PY@bf=\textbf\let\PY@it=\textit}
\@namedef{PY@tok@gp}{\let\PY@bf=\textbf\def\PY@tc##1{\textcolor[rgb]{0.00,0.00,0.50}{##1}}}
\@namedef{PY@tok@go}{\def\PY@tc##1{\textcolor[rgb]{0.44,0.44,0.44}{##1}}}
\@namedef{PY@tok@gt}{\def\PY@tc##1{\textcolor[rgb]{0.00,0.27,0.87}{##1}}}
\@namedef{PY@tok@err}{\def\PY@bc##1{{\setlength{\fboxsep}{\string -\fboxrule}\fcolorbox[rgb]{1.00,0.00,0.00}{1,1,1}{\strut ##1}}}}
\@namedef{PY@tok@kc}{\let\PY@bf=\textbf\def\PY@tc##1{\textcolor[rgb]{0.00,0.50,0.00}{##1}}}
\@namedef{PY@tok@kd}{\let\PY@bf=\textbf\def\PY@tc##1{\textcolor[rgb]{0.00,0.50,0.00}{##1}}}
\@namedef{PY@tok@kn}{\let\PY@bf=\textbf\def\PY@tc##1{\textcolor[rgb]{0.00,0.50,0.00}{##1}}}
\@namedef{PY@tok@kr}{\let\PY@bf=\textbf\def\PY@tc##1{\textcolor[rgb]{0.00,0.50,0.00}{##1}}}
\@namedef{PY@tok@bp}{\def\PY@tc##1{\textcolor[rgb]{0.00,0.50,0.00}{##1}}}
\@namedef{PY@tok@fm}{\def\PY@tc##1{\textcolor[rgb]{0.00,0.00,1.00}{##1}}}
\@namedef{PY@tok@vc}{\def\PY@tc##1{\textcolor[rgb]{0.10,0.09,0.49}{##1}}}
\@namedef{PY@tok@vg}{\def\PY@tc##1{\textcolor[rgb]{0.10,0.09,0.49}{##1}}}
\@namedef{PY@tok@vi}{\def\PY@tc##1{\textcolor[rgb]{0.10,0.09,0.49}{##1}}}
\@namedef{PY@tok@vm}{\def\PY@tc##1{\textcolor[rgb]{0.10,0.09,0.49}{##1}}}
\@namedef{PY@tok@sa}{\def\PY@tc##1{\textcolor[rgb]{0.73,0.13,0.13}{##1}}}
\@namedef{PY@tok@sb}{\def\PY@tc##1{\textcolor[rgb]{0.73,0.13,0.13}{##1}}}
\@namedef{PY@tok@sc}{\def\PY@tc##1{\textcolor[rgb]{0.73,0.13,0.13}{##1}}}
\@namedef{PY@tok@dl}{\def\PY@tc##1{\textcolor[rgb]{0.73,0.13,0.13}{##1}}}
\@namedef{PY@tok@s2}{\def\PY@tc##1{\textcolor[rgb]{0.73,0.13,0.13}{##1}}}
\@namedef{PY@tok@sh}{\def\PY@tc##1{\textcolor[rgb]{0.73,0.13,0.13}{##1}}}
\@namedef{PY@tok@s1}{\def\PY@tc##1{\textcolor[rgb]{0.73,0.13,0.13}{##1}}}
\@namedef{PY@tok@mb}{\def\PY@tc##1{\textcolor[rgb]{0.40,0.40,0.40}{##1}}}
\@namedef{PY@tok@mf}{\def\PY@tc##1{\textcolor[rgb]{0.40,0.40,0.40}{##1}}}
\@namedef{PY@tok@mh}{\def\PY@tc##1{\textcolor[rgb]{0.40,0.40,0.40}{##1}}}
\@namedef{PY@tok@mi}{\def\PY@tc##1{\textcolor[rgb]{0.40,0.40,0.40}{##1}}}
\@namedef{PY@tok@il}{\def\PY@tc##1{\textcolor[rgb]{0.40,0.40,0.40}{##1}}}
\@namedef{PY@tok@mo}{\def\PY@tc##1{\textcolor[rgb]{0.40,0.40,0.40}{##1}}}
\@namedef{PY@tok@ch}{\let\PY@it=\textit\def\PY@tc##1{\textcolor[rgb]{0.24,0.48,0.48}{##1}}}
\@namedef{PY@tok@cm}{\let\PY@it=\textit\def\PY@tc##1{\textcolor[rgb]{0.24,0.48,0.48}{##1}}}
\@namedef{PY@tok@cpf}{\let\PY@it=\textit\def\PY@tc##1{\textcolor[rgb]{0.24,0.48,0.48}{##1}}}
\@namedef{PY@tok@c1}{\let\PY@it=\textit\def\PY@tc##1{\textcolor[rgb]{0.24,0.48,0.48}{##1}}}
\@namedef{PY@tok@cs}{\let\PY@it=\textit\def\PY@tc##1{\textcolor[rgb]{0.24,0.48,0.48}{##1}}}

\def\PYZbs{\char`\\}
\def\PYZus{\char`\_}
\def\PYZob{\char`\{}
\def\PYZcb{\char`\}}
\def\PYZca{\char`\^}
\def\PYZam{\char`\&}
\def\PYZlt{\char`\<}
\def\PYZgt{\char`\>}
\def\PYZsh{\char`\#}
\def\PYZpc{\char`\%}
\def\PYZdl{\char`\$}
\def\PYZhy{\char`\-}
\def\PYZsq{\char`\'}
\def\PYZdq{\char`\"}
\def\PYZti{\char`\~}
% for compatibility with earlier versions
\def\PYZat{@}
\def\PYZlb{[}
\def\PYZrb{]}
\makeatother


    % For linebreaks inside Verbatim environment from package fancyvrb.
    \makeatletter
        \newbox\Wrappedcontinuationbox
        \newbox\Wrappedvisiblespacebox
        \newcommand*\Wrappedvisiblespace {\textcolor{red}{\textvisiblespace}}
        \newcommand*\Wrappedcontinuationsymbol {\textcolor{red}{\llap{\tiny$\m@th\hookrightarrow$}}}
        \newcommand*\Wrappedcontinuationindent {3ex }
        \newcommand*\Wrappedafterbreak {\kern\Wrappedcontinuationindent\copy\Wrappedcontinuationbox}
        % Take advantage of the already applied Pygments mark-up to insert
        % potential linebreaks for TeX processing.
        %        {, <, #, %, $, ' and ": go to next line.
        %        _, }, ^, &, >, - and ~: stay at end of broken line.
        % Use of \textquotesingle for straight quote.
        \newcommand*\Wrappedbreaksatspecials {%
            \def\PYGZus{\discretionary{\char`\_}{\Wrappedafterbreak}{\char`\_}}%
            \def\PYGZob{\discretionary{}{\Wrappedafterbreak\char`\{}{\char`\{}}%
            \def\PYGZcb{\discretionary{\char`\}}{\Wrappedafterbreak}{\char`\}}}%
            \def\PYGZca{\discretionary{\char`\^}{\Wrappedafterbreak}{\char`\^}}%
            \def\PYGZam{\discretionary{\char`\&}{\Wrappedafterbreak}{\char`\&}}%
            \def\PYGZlt{\discretionary{}{\Wrappedafterbreak\char`\<}{\char`\<}}%
            \def\PYGZgt{\discretionary{\char`\>}{\Wrappedafterbreak}{\char`\>}}%
            \def\PYGZsh{\discretionary{}{\Wrappedafterbreak\char`\#}{\char`\#}}%
            \def\PYGZpc{\discretionary{}{\Wrappedafterbreak\char`\%}{\char`\%}}%
            \def\PYGZdl{\discretionary{}{\Wrappedafterbreak\char`\$}{\char`\$}}%
            \def\PYGZhy{\discretionary{\char`\-}{\Wrappedafterbreak}{\char`\-}}%
            \def\PYGZsq{\discretionary{}{\Wrappedafterbreak\textquotesingle}{\textquotesingle}}%
            \def\PYGZdq{\discretionary{}{\Wrappedafterbreak\char`\"}{\char`\"}}%
            \def\PYGZti{\discretionary{\char`\~}{\Wrappedafterbreak}{\char`\~}}%
        }
        % Some characters . , ; ? ! / are not pygmentized.
        % This macro makes them "active" and they will insert potential linebreaks
        \newcommand*\Wrappedbreaksatpunct {%
            \lccode`\~`\.\lowercase{\def~}{\discretionary{\hbox{\char`\.}}{\Wrappedafterbreak}{\hbox{\char`\.}}}%
            \lccode`\~`\,\lowercase{\def~}{\discretionary{\hbox{\char`\,}}{\Wrappedafterbreak}{\hbox{\char`\,}}}%
            \lccode`\~`\;\lowercase{\def~}{\discretionary{\hbox{\char`\;}}{\Wrappedafterbreak}{\hbox{\char`\;}}}%
            \lccode`\~`\:\lowercase{\def~}{\discretionary{\hbox{\char`\:}}{\Wrappedafterbreak}{\hbox{\char`\:}}}%
            \lccode`\~`\?\lowercase{\def~}{\discretionary{\hbox{\char`\?}}{\Wrappedafterbreak}{\hbox{\char`\?}}}%
            \lccode`\~`\!\lowercase{\def~}{\discretionary{\hbox{\char`\!}}{\Wrappedafterbreak}{\hbox{\char`\!}}}%
            \lccode`\~`\/\lowercase{\def~}{\discretionary{\hbox{\char`\/}}{\Wrappedafterbreak}{\hbox{\char`\/}}}%
            \catcode`\.\active
            \catcode`\,\active
            \catcode`\;\active
            \catcode`\:\active
            \catcode`\?\active
            \catcode`\!\active
            \catcode`\/\active
            \lccode`\~`\~
        }
    \makeatother

    \let\OriginalVerbatim=\Verbatim
    \makeatletter
    \renewcommand{\Verbatim}[1][1]{%
        %\parskip\z@skip
        \sbox\Wrappedcontinuationbox {\Wrappedcontinuationsymbol}%
        \sbox\Wrappedvisiblespacebox {\FV@SetupFont\Wrappedvisiblespace}%
        \def\FancyVerbFormatLine ##1{\hsize\linewidth
            \vtop{\raggedright\hyphenpenalty\z@\exhyphenpenalty\z@
                \doublehyphendemerits\z@\finalhyphendemerits\z@
                \strut ##1\strut}%
        }%
        % If the linebreak is at a space, the latter will be displayed as visible
        % space at end of first line, and a continuation symbol starts next line.
        % Stretch/shrink are however usually zero for typewriter font.
        \def\FV@Space {%
            \nobreak\hskip\z@ plus\fontdimen3\font minus\fontdimen4\font
            \discretionary{\copy\Wrappedvisiblespacebox}{\Wrappedafterbreak}
            {\kern\fontdimen2\font}%
        }%

        % Allow breaks at special characters using \PYG... macros.
        \Wrappedbreaksatspecials
        % Breaks at punctuation characters . , ; ? ! and / need catcode=\active
        \OriginalVerbatim[#1,codes*=\Wrappedbreaksatpunct]%
    }
    \makeatother

    % Exact colors from NB
    \definecolor{incolor}{HTML}{303F9F}
    \definecolor{outcolor}{HTML}{D84315}
    \definecolor{cellborder}{HTML}{CFCFCF}
    \definecolor{cellbackground}{HTML}{F7F7F7}

    % prompt
    \makeatletter
    \newcommand{\boxspacing}{\kern\kvtcb@left@rule\kern\kvtcb@boxsep}
    \makeatother
    \newcommand{\prompt}[4]{
        {\ttfamily\llap{{\color{#2}[#3]:\hspace{3pt}#4}}\vspace{-\baselineskip}}
    }
    

    
    % Prevent overflowing lines due to hard-to-break entities
    \sloppy
    % Setup hyperref package
    \hypersetup{
      breaklinks=true,  % so long urls are correctly broken across lines
      colorlinks=true,
      urlcolor=urlcolor,
      linkcolor=linkcolor,
      citecolor=citecolor,
      }
    % Slightly bigger margins than the latex defaults
    
    \geometry{verbose,tmargin=1in,bmargin=1in,lmargin=1in,rmargin=1in}
    
    

\begin{document}
    
    \maketitle
    
    

    
    \hypertarget{analyze-data-split-using-80-20-balancing-score-jensen-shannon-distance-appearing-families}{%
\section{Analyze data split using 80-20 Balancing score, Jensen-Shannon
distance, \% Appearing
families}\label{analyze-data-split-using-80-20-balancing-score-jensen-shannon-distance-appearing-families}}

    \hypertarget{setup-the-data-frame-and-define-the-functions}{%
\subsection{Setup the data frame and define the
functions}\label{setup-the-data-frame-and-define-the-functions}}

    \begin{tcolorbox}[breakable, size=fbox, boxrule=1pt, pad at break*=1mm,colback=cellbackground, colframe=cellborder]
\prompt{In}{incolor}{1}{\boxspacing}
\begin{Verbatim}[commandchars=\\\{\}]
\PY{k+kn}{from} \PY{n+nn}{utils}\PY{n+nn}{.}\PY{n+nn}{best\PYZus{}split\PYZus{}utils} \PY{k+kn}{import} \PY{o}{*}
\PY{k+kn}{import} \PY{n+nn}{seaborn} \PY{k}{as} \PY{n+nn}{sns}
\PY{k+kn}{import} \PY{n+nn}{matplotlib}\PY{n+nn}{.}\PY{n+nn}{pyplot} \PY{k}{as} \PY{n+nn}{plt}
\end{Verbatim}
\end{tcolorbox}

    \begin{tcolorbox}[breakable, size=fbox, boxrule=1pt, pad at break*=1mm,colback=cellbackground, colframe=cellborder]
\prompt{In}{incolor}{2}{\boxspacing}
\begin{Verbatim}[commandchars=\\\{\}]
\PY{c+c1}{\PYZsh{} Get the merged malware data}
\PY{n}{df} \PY{o}{=} \PY{n}{pd}\PY{o}{.}\PY{n}{read\PYZus{}csv}\PY{p}{(}\PY{l+s+s2}{\PYZdq{}}\PY{l+s+s2}{vt\PYZus{}reports/merge.csv}\PY{l+s+s2}{\PYZdq{}}\PY{p}{)}
\PY{n}{df}\PY{o}{.}\PY{n}{head}\PY{p}{(}\PY{p}{)}
\end{Verbatim}
\end{tcolorbox}

            \begin{tcolorbox}[breakable, size=fbox, boxrule=.5pt, pad at break*=1mm, opacityfill=0]
\prompt{Out}{outcolor}{2}{\boxspacing}
\begin{Verbatim}[commandchars=\\\{\}]
                                              sha256  first\_submission\_date  \textbackslash{}
0  98f8e26e12b978102fa39c197f300ebe5fe535617737d5{\ldots}             1630575593
1  7b2999ffadbc3b5b5c5e94145ca4e2f8de66ac1e3ddd52{\ldots}             1629375559
2  e7569d494fe00be04ef6c9fcc5e54720c0df623b08e79d{\ldots}             1362057319
3  1ed60c04f572b6acb9f64c31db55ef5c6b5465bd4da1eb{\ldots}             1630624233
4  4c4aaff20a57213d9a786e56ad22f1eaa94694a2f1042b{\ldots}             1592186154

     family
0     tnega
1    quasar
2     pasta
3    cjishu
4  kingsoft
\end{Verbatim}
\end{tcolorbox}
        
    \begin{tcolorbox}[breakable, size=fbox, boxrule=1pt, pad at break*=1mm,colback=cellbackground, colframe=cellborder]
\prompt{In}{incolor}{3}{\boxspacing}
\begin{Verbatim}[commandchars=\\\{\}]
\PY{n}{fsd} \PY{o}{=} \PY{l+s+s2}{\PYZdq{}}\PY{l+s+s2}{first\PYZus{}submission\PYZus{}date}\PY{l+s+s2}{\PYZdq{}}
\PY{c+c1}{\PYZsh{} Convert the timestamps to datetime format}
\PY{n}{df\PYZus{}dt} \PY{o}{=} \PY{n}{df}\PY{o}{.}\PY{n}{copy}\PY{p}{(}\PY{p}{)}
\PY{n}{df\PYZus{}dt}\PY{p}{[}\PY{n}{fsd}\PY{p}{]} \PY{o}{=} \PY{n}{df\PYZus{}dt}\PY{p}{[}\PY{n}{fsd}\PY{p}{]}\PY{o}{.}\PY{n}{apply}\PY{p}{(}\PY{k}{lambda} \PY{n}{t}\PY{p}{:} \PY{n}{pd}\PY{o}{.}\PY{n}{to\PYZus{}datetime}\PY{p}{(}\PY{n}{t}\PY{p}{,} \PY{n}{unit}\PY{o}{=}\PY{l+s+s1}{\PYZsq{}}\PY{l+s+s1}{s}\PY{l+s+s1}{\PYZsq{}}\PY{p}{)}\PY{p}{)}
\end{Verbatim}
\end{tcolorbox}

    \begin{tcolorbox}[breakable, size=fbox, boxrule=1pt, pad at break*=1mm,colback=cellbackground, colframe=cellborder]
\prompt{In}{incolor}{4}{\boxspacing}
\begin{Verbatim}[commandchars=\\\{\}]
\PY{k+kn}{from} \PY{n+nn}{typing} \PY{k+kn}{import} \PY{n}{Callable}
\PY{k+kn}{from} \PY{n+nn}{scipy}\PY{n+nn}{.}\PY{n+nn}{spatial}\PY{n+nn}{.}\PY{n+nn}{distance} \PY{k+kn}{import} \PY{n}{jensenshannon}
\PY{k}{def} \PY{n+nf}{compute\PYZus{}scores}\PY{p}{(}\PY{n}{df}\PY{p}{:} \PY{n}{pd}\PY{o}{.}\PY{n}{DataFrame}\PY{p}{,} \PY{n}{ref\PYZus{}df}\PY{p}{:} \PY{n}{pd}\PY{o}{.}\PY{n}{DataFrame}\PY{p}{,} \PY{n}{date\PYZus{}split}\PY{p}{:} \PY{n}{pd}\PY{o}{.}\PY{n}{Timestamp}\PY{p}{)}\PY{p}{:}
\PY{+w}{    }\PY{l+s+sd}{\PYZdq{}\PYZdq{}\PYZdq{}}
\PY{l+s+sd}{    Compute the scores given a dataset and a timestamp as data split:}
\PY{l+s+sd}{    Jensen\PYZhy{}Shannon score, Train\PYZhy{}Test balancing, \PYZpc{} Appearing families in testing set}
\PY{l+s+sd}{    \PYZdq{}\PYZdq{}\PYZdq{}}
    \PY{c+c1}{\PYZsh{} JS}
    \PY{n}{df\PYZus{}train\PYZus{}all} \PY{o}{=} \PY{n}{split\PYZus{}and\PYZus{}group}\PY{p}{(}\PY{n}{src\PYZus{}df}\PY{o}{=}\PY{n}{df}\PY{p}{,} \PY{n}{split\PYZus{}condition}\PY{o}{=}\PY{n}{df}\PY{p}{[}\PY{n}{fsd}\PY{p}{]} \PY{o}{\PYZlt{}} \PY{n}{date\PYZus{}split}\PY{p}{,} \PY{n}{ref\PYZus{}df}\PY{o}{=}\PY{n}{ref\PYZus{}df}\PY{p}{)}
    \PY{n}{df\PYZus{}test\PYZus{}all} \PY{o}{=} \PY{n}{split\PYZus{}and\PYZus{}group}\PY{p}{(}\PY{n}{src\PYZus{}df}\PY{o}{=}\PY{n}{df}\PY{p}{,} \PY{n}{split\PYZus{}condition}\PY{o}{=}\PY{n}{df}\PY{p}{[}\PY{n}{fsd}\PY{p}{]} \PY{o}{\PYZgt{}}\PY{o}{=} \PY{n}{date\PYZus{}split}\PY{p}{,} \PY{n}{ref\PYZus{}df}\PY{o}{=}\PY{n}{ref\PYZus{}df}\PY{p}{)}
    \PY{n}{js} \PY{o}{=} \PY{n}{jensenshannon}\PY{p}{(}\PY{n}{np}\PY{o}{.}\PY{n}{array}\PY{p}{(}\PY{n}{df\PYZus{}train\PYZus{}all}\PY{p}{[}\PY{l+s+s2}{\PYZdq{}}\PY{l+s+s2}{count}\PY{l+s+s2}{\PYZdq{}}\PY{p}{]}\PY{p}{)}\PY{p}{,} \PY{n}{np}\PY{o}{.}\PY{n}{array}\PY{p}{(}\PY{n}{df\PYZus{}test\PYZus{}all}\PY{p}{[}\PY{l+s+s2}{\PYZdq{}}\PY{l+s+s2}{count}\PY{l+s+s2}{\PYZdq{}}\PY{p}{]}\PY{p}{)}\PY{p}{)}

    \PY{c+c1}{\PYZsh{} Train\PYZhy{}Test balancing: this score increases as the training test length}
    \PY{c+c1}{\PYZsh{} in \PYZpc{} is approaching 80\PYZpc{} of the samples}
    \PY{n}{train\PYZus{}prop} \PY{o}{=} \PY{n+nb}{len}\PY{p}{(}\PY{n}{df}\PY{p}{[}\PY{n}{df}\PY{p}{[}\PY{n}{fsd}\PY{p}{]} \PY{o}{\PYZlt{}} \PY{n}{date\PYZus{}split}\PY{p}{]}\PY{p}{)} \PY{o}{/} \PY{n+nb}{len}\PY{p}{(}\PY{n}{ref\PYZus{}df}\PY{p}{)}
    \PY{n}{bs} \PY{o}{=} \PY{l+m+mi}{1} \PY{o}{\PYZhy{}} \PY{n}{np}\PY{o}{.}\PY{n}{abs}\PY{p}{(}\PY{n}{train\PYZus{}prop} \PY{o}{\PYZhy{}} \PY{l+m+mf}{0.8}\PY{p}{)} \PY{o}{/} \PY{l+m+mf}{0.8}

    \PY{c+c1}{\PYZsh{} \PYZpc{} Appearing families in testing set}
    \PY{n}{df\PYZus{}train\PYZus{}nonzero} \PY{o}{=} \PY{n}{split\PYZus{}and\PYZus{}group\PYZus{}nonzero}\PY{p}{(}\PY{n}{src\PYZus{}df}\PY{o}{=}\PY{n}{df}\PY{p}{,} \PY{n}{split\PYZus{}condition}\PY{o}{=}\PY{n}{df}\PY{p}{[}\PY{n}{fsd}\PY{p}{]} \PY{o}{\PYZlt{}} \PY{n}{date\PYZus{}split}\PY{p}{)}
    \PY{n}{df\PYZus{}test\PYZus{}nonzero} \PY{o}{=} \PY{n}{split\PYZus{}and\PYZus{}group\PYZus{}nonzero}\PY{p}{(}\PY{n}{src\PYZus{}df}\PY{o}{=}\PY{n}{df}\PY{p}{,} \PY{n}{split\PYZus{}condition}\PY{o}{=}\PY{n}{df}\PY{p}{[}\PY{n}{fsd}\PY{p}{]} \PY{o}{\PYZgt{}}\PY{o}{=} \PY{n}{date\PYZus{}split}\PY{p}{)}

    \PY{n}{test\PYZus{}families} \PY{o}{=} \PY{n}{df\PYZus{}test\PYZus{}nonzero}\PY{p}{[}\PY{l+s+s2}{\PYZdq{}}\PY{l+s+s2}{family}\PY{l+s+s2}{\PYZdq{}}\PY{p}{]}\PY{o}{.}\PY{n}{unique}\PY{p}{(}\PY{p}{)}
    \PY{n}{af} \PY{o}{=} \PY{p}{(}\PY{p}{(}\PY{n+nb}{len}\PY{p}{(}\PY{n}{test\PYZus{}families}\PY{p}{)} \PY{o}{\PYZhy{}} \PY{n+nb}{len}\PY{p}{(}\PY{n}{np}\PY{o}{.}\PY{n}{intersect1d}\PY{p}{(}\PY{n}{df\PYZus{}train\PYZus{}nonzero}\PY{p}{[}\PY{l+s+s2}{\PYZdq{}}\PY{l+s+s2}{family}\PY{l+s+s2}{\PYZdq{}}\PY{p}{]}\PY{o}{.}\PY{n}{unique}\PY{p}{(}\PY{p}{)}\PY{p}{,} \PY{n}{test\PYZus{}families}\PY{p}{)}\PY{p}{)}\PY{p}{)} \PY{o}{/}
          \PY{n+nb}{len}\PY{p}{(}\PY{n}{ref\PYZus{}df}\PY{p}{[}\PY{l+s+s2}{\PYZdq{}}\PY{l+s+s2}{family}\PY{l+s+s2}{\PYZdq{}}\PY{p}{]}\PY{o}{.}\PY{n}{unique}\PY{p}{(}\PY{p}{)}\PY{p}{)}\PY{p}{)}

    \PY{k}{return} \PY{p}{\PYZob{}}\PY{l+s+s2}{\PYZdq{}}\PY{l+s+s2}{js}\PY{l+s+s2}{\PYZdq{}}\PY{p}{:} \PY{n}{js}\PY{p}{,} \PY{l+s+s2}{\PYZdq{}}\PY{l+s+s2}{bs}\PY{l+s+s2}{\PYZdq{}}\PY{p}{:} \PY{n}{bs}\PY{p}{,} \PY{l+s+s2}{\PYZdq{}}\PY{l+s+s2}{af}\PY{l+s+s2}{\PYZdq{}}\PY{p}{:} \PY{n}{af}\PY{p}{\PYZcb{}}
\end{Verbatim}
\end{tcolorbox}

    \hypertarget{compute-the-scores-using-the-first-day-of-each-month-as-data-splits}{%
\subsection{Compute the scores using the first day of each month as data
splits}\label{compute-the-scores-using-the-first-day-of-each-month-as-data-splits}}

    \begin{tcolorbox}[breakable, size=fbox, boxrule=1pt, pad at break*=1mm,colback=cellbackground, colframe=cellborder]
\prompt{In}{incolor}{5}{\boxspacing}
\begin{Verbatim}[commandchars=\\\{\}]
\PY{c+c1}{\PYZsh{} Min and maximum dates}
\PY{n}{date\PYZus{}min} \PY{o}{=} \PY{n}{df\PYZus{}dt}\PY{p}{[}\PY{n}{fsd}\PY{p}{]}\PY{o}{.}\PY{n}{min}\PY{p}{(}\PY{p}{)}
\PY{n}{date\PYZus{}max} \PY{o}{=} \PY{n}{df\PYZus{}dt}\PY{p}{[}\PY{n}{fsd}\PY{p}{]}\PY{o}{.}\PY{n}{max}\PY{p}{(}\PY{p}{)}

\PY{n}{date\PYZus{}min\PYZus{}n} \PY{o}{=} \PY{n}{pd}\PY{o}{.}\PY{n}{Timestamp}\PY{p}{(}\PY{l+s+sa}{f}\PY{l+s+s2}{\PYZdq{}}\PY{l+s+si}{\PYZob{}}\PY{n}{date\PYZus{}min}\PY{o}{.}\PY{n}{year}\PY{l+s+si}{\PYZcb{}}\PY{l+s+s2}{\PYZhy{}}\PY{l+s+si}{\PYZob{}}\PY{n}{date\PYZus{}min}\PY{o}{.}\PY{n}{month}\PY{l+s+si}{\PYZcb{}}\PY{l+s+s2}{\PYZhy{}}\PY{l+s+si}{\PYZob{}}\PY{n}{date\PYZus{}min}\PY{o}{.}\PY{n}{day}\PY{l+s+si}{\PYZcb{}}\PY{l+s+s2}{\PYZdq{}}\PY{p}{)}
\PY{n}{date\PYZus{}max\PYZus{}n} \PY{o}{=} \PY{n}{pd}\PY{o}{.}\PY{n}{Timestamp}\PY{p}{(}\PY{l+s+sa}{f}\PY{l+s+s2}{\PYZdq{}}\PY{l+s+si}{\PYZob{}}\PY{n}{date\PYZus{}max}\PY{o}{.}\PY{n}{year}\PY{l+s+si}{\PYZcb{}}\PY{l+s+s2}{\PYZhy{}}\PY{l+s+si}{\PYZob{}}\PY{n}{date\PYZus{}max}\PY{o}{.}\PY{n}{month}\PY{l+s+si}{\PYZcb{}}\PY{l+s+s2}{\PYZhy{}}\PY{l+s+si}{\PYZob{}}\PY{n}{date\PYZus{}max}\PY{o}{.}\PY{n}{day}\PY{l+s+si}{\PYZcb{}}\PY{l+s+s2}{\PYZdq{}}\PY{p}{)}

\PY{c+c1}{\PYZsh{} Create 1\PYZhy{}month equidistant splits}
\PY{c+c1}{\PYZsh{} \PYZdq{}MS\PYZdq{}: use the Start of each Month from the minimum date to the maximum}
\PY{n}{date\PYZus{}splits} \PY{o}{=} \PY{n}{pd}\PY{o}{.}\PY{n}{date\PYZus{}range}\PY{p}{(}\PY{n}{start}\PY{o}{=}\PY{n}{date\PYZus{}min\PYZus{}n}\PY{p}{,} \PY{n}{end}\PY{o}{=}\PY{n}{date\PYZus{}max\PYZus{}n}\PY{p}{,} \PY{n}{freq}\PY{o}{=}\PY{l+s+s2}{\PYZdq{}}\PY{l+s+s2}{MS}\PY{l+s+s2}{\PYZdq{}}\PY{p}{)}\PY{o}{.}\PY{n}{tolist}\PY{p}{(}\PY{p}{)}
\end{Verbatim}
\end{tcolorbox}

    \begin{tcolorbox}[breakable, size=fbox, boxrule=1pt, pad at break*=1mm,colback=cellbackground, colframe=cellborder]
\prompt{In}{incolor}{6}{\boxspacing}
\begin{Verbatim}[commandchars=\\\{\}]
\PY{n}{df\PYZus{}scores}\PY{p}{,} \PY{n}{df\PYZus{}ref\PYZus{}scores} \PY{o}{=} \PY{n}{df\PYZus{}dt}\PY{o}{.}\PY{n}{copy}\PY{p}{(}\PY{p}{)}\PY{p}{,} \PY{n}{df\PYZus{}dt}\PY{o}{.}\PY{n}{copy}\PY{p}{(}\PY{p}{)}
\PY{n}{js\PYZus{}scores}\PY{p}{,} \PY{n}{perc\PYZus{}app\PYZus{}families}\PY{p}{,} \PY{n}{balance\PYZus{}scores} \PY{o}{=} \PY{p}{[}\PY{p}{]}\PY{p}{,} \PY{p}{[}\PY{p}{]}\PY{p}{,} \PY{p}{[}\PY{p}{]}
\PY{k}{for} \PY{n}{date\PYZus{}split} \PY{o+ow}{in} \PY{n}{date\PYZus{}splits}\PY{p}{:}
    \PY{n}{scores} \PY{o}{=} \PY{n}{compute\PYZus{}scores}\PY{p}{(}\PY{n}{df}\PY{o}{=}\PY{n}{df\PYZus{}scores}\PY{p}{,} \PY{n}{ref\PYZus{}df}\PY{o}{=}\PY{n}{df\PYZus{}ref\PYZus{}scores}\PY{p}{,} \PY{n}{date\PYZus{}split}\PY{o}{=}\PY{n}{date\PYZus{}split}\PY{p}{)}
    \PY{n}{js\PYZus{}scores}\PY{o}{.}\PY{n}{append}\PY{p}{(}\PY{n}{scores}\PY{p}{[}\PY{l+s+s2}{\PYZdq{}}\PY{l+s+s2}{js}\PY{l+s+s2}{\PYZdq{}}\PY{p}{]}\PY{p}{)}
    \PY{n}{perc\PYZus{}app\PYZus{}families}\PY{o}{.}\PY{n}{append}\PY{p}{(}\PY{n}{scores}\PY{p}{[}\PY{l+s+s2}{\PYZdq{}}\PY{l+s+s2}{af}\PY{l+s+s2}{\PYZdq{}}\PY{p}{]}\PY{p}{)}
    \PY{n}{balance\PYZus{}scores}\PY{o}{.}\PY{n}{append}\PY{p}{(}\PY{n}{scores}\PY{p}{[}\PY{l+s+s2}{\PYZdq{}}\PY{l+s+s2}{bs}\PY{l+s+s2}{\PYZdq{}}\PY{p}{]}\PY{p}{)}
\end{Verbatim}
\end{tcolorbox}

    \hypertarget{plot-the-80-20-balancing-score-between-train-and-test-set}{%
\subsection{Plot the 80-20 Balancing score between train and test
set}\label{plot-the-80-20-balancing-score-between-train-and-test-set}}

As shown below, the highest balancing score is achieved using 2021-12-01
as the timestamp split. However, very few number of families are
introduced in the test set (9 families, 1.34\%). We further investigate
the data on the latest years where there's both an increase and peak of
\(BS\) to see if we can choose a point where the number of appearing
families is higher.

    \begin{tcolorbox}[breakable, size=fbox, boxrule=1pt, pad at break*=1mm,colback=cellbackground, colframe=cellborder]
\prompt{In}{incolor}{7}{\boxspacing}
\begin{Verbatim}[commandchars=\\\{\}]
\PY{n}{max\PYZus{}bs}\PY{p}{,} \PY{n}{idx\PYZus{}max\PYZus{}bs} \PY{o}{=} \PY{n}{np}\PY{o}{.}\PY{n}{max}\PY{p}{(}\PY{n}{balance\PYZus{}scores}\PY{p}{)}\PY{p}{,} \PY{n}{np}\PY{o}{.}\PY{n}{argmax}\PY{p}{(}\PY{n}{balance\PYZus{}scores}\PY{p}{)}

\PY{n}{max\PYZus{}bs\PYZus{}split} \PY{o}{=} \PY{n}{date\PYZus{}splits}\PY{p}{[}\PY{n}{idx\PYZus{}max\PYZus{}bs}\PY{p}{]}
\PY{n+nb}{print}\PY{p}{(}\PY{l+s+sa}{f}\PY{l+s+s2}{\PYZdq{}}\PY{l+s+s2}{Max balance score }\PY{l+s+si}{\PYZob{}}\PY{n}{max\PYZus{}bs}\PY{l+s+si}{\PYZcb{}}\PY{l+s+s2}{\PYZdq{}}\PY{p}{)}
\PY{n}{print\PYZus{}statistics}\PY{p}{(}\PY{n}{df\PYZus{}scores}\PY{p}{,} \PY{n}{max\PYZus{}bs\PYZus{}split}\PY{p}{,} \PY{l+s+sa}{f}\PY{l+s+s2}{\PYZdq{}}\PY{l+s+s2}{Best split based on Balance score at: }\PY{l+s+si}{\PYZob{}}\PY{n}{max\PYZus{}bs\PYZus{}split}\PY{l+s+si}{\PYZcb{}}\PY{l+s+s2}{\PYZdq{}}\PY{p}{)}

\PY{n}{df\PYZus{}bs} \PY{o}{=} \PY{n}{pd}\PY{o}{.}\PY{n}{DataFrame}\PY{p}{(}\PY{p}{\PYZob{}}\PY{l+s+s2}{\PYZdq{}}\PY{l+s+s2}{split}\PY{l+s+s2}{\PYZdq{}}\PY{p}{:} \PY{n}{date\PYZus{}splits}\PY{p}{,} \PY{l+s+s2}{\PYZdq{}}\PY{l+s+s2}{score}\PY{l+s+s2}{\PYZdq{}}\PY{p}{:} \PY{n}{balance\PYZus{}scores}\PY{p}{\PYZcb{}}\PY{p}{)}
\PY{n}{plt}\PY{o}{.}\PY{n}{figure}\PY{p}{(}\PY{n}{figsize}\PY{o}{=}\PY{p}{(}\PY{l+m+mi}{10}\PY{p}{,} \PY{l+m+mi}{7}\PY{p}{)}\PY{p}{)}
\PY{n}{ax} \PY{o}{=} \PY{n}{sns}\PY{o}{.}\PY{n}{barplot}\PY{p}{(}\PY{n}{data}\PY{o}{=}\PY{n}{df\PYZus{}bs}\PY{p}{,} \PY{n}{x}\PY{o}{=}\PY{l+s+s2}{\PYZdq{}}\PY{l+s+s2}{split}\PY{l+s+s2}{\PYZdq{}}\PY{p}{,} \PY{n}{y}\PY{o}{=}\PY{l+s+s2}{\PYZdq{}}\PY{l+s+s2}{score}\PY{l+s+s2}{\PYZdq{}}\PY{p}{)}

\PY{n}{plt}\PY{o}{.}\PY{n}{title}\PY{p}{(}\PY{l+s+s2}{\PYZdq{}}\PY{l+s+s2}{80\PYZhy{}20 training/testing Balance Score}\PY{l+s+s2}{\PYZdq{}}\PY{p}{)}
\PY{n}{plt}\PY{o}{.}\PY{n}{xticks}\PY{p}{(}\PY{p}{[}\PY{p}{]}\PY{p}{)}
\PY{n}{plt}\PY{o}{.}\PY{n}{show}\PY{p}{(}\PY{p}{)}
\end{Verbatim}
\end{tcolorbox}

    \begin{Verbatim}[commandchars=\\\{\}]
Max balance score 0.8575
------------------------------------------------------------------
Report: Best split based on Balance score at: 2021-12-01 00:00:00
        Training set length: 45962, (68.6\%)
        Testing set length: 21038, (31.4\%)
        Num families in training: 661
        Num families in testing: 611
        Common families: 602
        Families in training but not in testing: 59 (8.81\%)
        Families in testing but not in training: 9 (1.34\%)
    \end{Verbatim}

    \begin{center}
    \adjustimage{max size={0.9\linewidth}{0.9\paperheight}}{best_split_files/best_split_10_1.png}
    \end{center}
    { \hspace*{\fill} \\}
    
    \hypertarget{score-focus}{%
\subsection{Score focus}\label{score-focus}}

From later on, the analysis will focus on the latest years, where
there's both an increase and peak of \(BS\). Splits at and after
2021-07-01 are considered, where the \(BS > 0.5\).

\hypertarget{observations}{%
\subsubsection{Observations}\label{observations}}

\begin{itemize}
\tightlist
\item
  Choosing \(BS > 0.5\) implies initiating with a basis comprising
  40.18\% of the training set length;
\item
  In 2021-08 lots of samples are submitted: at split 2021-08-01 the
  training length in percentage is 40.92\%, and at 2021-09-01 it's
  60.93\%. In the first case 74 (11.04\%) of new families are
  introduced, 16 in the second (2.39\%);
\item
  At 2021-09-01 16 families are introduced in the test set as opposed to
  9 of the subsequent three splits, while having 60.93\% as training set
  length;
\item
  From 2021-10-01 to 2021-12-01 (included), the number of appearing
  families remain stable at 9 (as described before), while the training
  set length goes from 67.84\% to 68.6\%;
\item
  In the last three splits (from 2022-01-01) there isn't any new
  appearing family. Those splits are not interesting to study family
  drift, so they will be not considered;
\end{itemize}

    \begin{tcolorbox}[breakable, size=fbox, boxrule=1pt, pad at break*=1mm,colback=cellbackground, colframe=cellborder]
\prompt{In}{incolor}{8}{\boxspacing}
\begin{Verbatim}[commandchars=\\\{\}]
\PY{n}{plt}\PY{o}{.}\PY{n}{figure}\PY{p}{(}\PY{n}{figsize}\PY{o}{=}\PY{p}{(}\PY{l+m+mi}{10}\PY{p}{,} \PY{l+m+mi}{7}\PY{p}{)}\PY{p}{)}

\PY{n}{df\PYZus{}bs\PYZus{}focus} \PY{o}{=} \PY{n}{df\PYZus{}bs}\PY{p}{[}\PY{n}{df\PYZus{}bs}\PY{p}{[}\PY{l+s+s2}{\PYZdq{}}\PY{l+s+s2}{score}\PY{l+s+s2}{\PYZdq{}}\PY{p}{]} \PY{o}{\PYZgt{}} \PY{l+m+mf}{0.5}\PY{p}{]}
\PY{n}{t\PYZus{}focus} \PY{o}{=} \PY{n}{df\PYZus{}bs\PYZus{}focus}\PY{p}{[}\PY{l+s+s2}{\PYZdq{}}\PY{l+s+s2}{split}\PY{l+s+s2}{\PYZdq{}}\PY{p}{]}\PY{o}{.}\PY{n}{min}\PY{p}{(}\PY{p}{)}

\PY{n}{ax} \PY{o}{=} \PY{n}{sns}\PY{o}{.}\PY{n}{barplot}\PY{p}{(}\PY{n}{data}\PY{o}{=}\PY{n}{df\PYZus{}bs\PYZus{}focus}\PY{p}{,} \PY{n}{x}\PY{o}{=}\PY{l+s+s2}{\PYZdq{}}\PY{l+s+s2}{split}\PY{l+s+s2}{\PYZdq{}}\PY{p}{,} \PY{n}{y}\PY{o}{=}\PY{l+s+s2}{\PYZdq{}}\PY{l+s+s2}{score}\PY{l+s+s2}{\PYZdq{}}\PY{p}{)}
\PY{n}{plt}\PY{o}{.}\PY{n}{title}\PY{p}{(}\PY{l+s+sa}{f}\PY{l+s+s2}{\PYZdq{}}\PY{l+s+s2}{80\PYZhy{}20 training/testing Balance Score focus on latest years (BS \PYZgt{}0.5)}\PY{l+s+s2}{\PYZdq{}}\PY{p}{)}
\PY{n}{plt}\PY{o}{.}\PY{n}{xticks}\PY{p}{(}\PY{n}{rotation}\PY{o}{=}\PY{l+m+mi}{90}\PY{p}{)}
\PY{n}{plt}\PY{o}{.}\PY{n}{show}\PY{p}{(}\PY{p}{)}

\PY{k}{for} \PY{n}{split} \PY{o+ow}{in} \PY{n}{df\PYZus{}bs\PYZus{}focus}\PY{p}{[}\PY{l+s+s2}{\PYZdq{}}\PY{l+s+s2}{split}\PY{l+s+s2}{\PYZdq{}}\PY{p}{]}\PY{p}{:}
    \PY{n}{print\PYZus{}statistics}\PY{p}{(}\PY{n}{df\PYZus{}scores}\PY{p}{,} \PY{n}{split}\PY{p}{,} \PY{l+s+sa}{f}\PY{l+s+s2}{\PYZdq{}}\PY{l+s+si}{\PYZob{}}\PY{n}{split}\PY{l+s+si}{\PYZcb{}}\PY{l+s+s2}{, BS: }\PY{l+s+si}{\PYZob{}}\PY{n}{df\PYZus{}bs\PYZus{}focus}\PY{p}{[}\PY{n}{df\PYZus{}bs\PYZus{}focus}\PY{p}{[}\PY{l+s+s1}{\PYZsq{}}\PY{l+s+s1}{split}\PY{l+s+s1}{\PYZsq{}}\PY{p}{]}\PY{+w}{ }\PY{o}{==}\PY{+w}{ }\PY{n}{split}\PY{p}{]}\PY{p}{[}\PY{l+s+s1}{\PYZsq{}}\PY{l+s+s1}{score}\PY{l+s+s1}{\PYZsq{}}\PY{p}{]}\PY{o}{.}\PY{n}{iloc}\PY{p}{[}\PY{l+m+mi}{0}\PY{p}{]}\PY{l+s+si}{\PYZcb{}}\PY{l+s+s2}{\PYZdq{}}\PY{p}{)}
\end{Verbatim}
\end{tcolorbox}

    \begin{center}
    \adjustimage{max size={0.9\linewidth}{0.9\paperheight}}{best_split_files/best_split_12_0.png}
    \end{center}
    { \hspace*{\fill} \\}
    
    \begin{Verbatim}[commandchars=\\\{\}]
------------------------------------------------------------------
Report: 2021-07-01 00:00:00, BS: 0.502276119402985
        Training set length: 26922, (40.18\%)
        Testing set length: 40078, (59.82\%)
        Num families in training: 595
        Num families in testing: 650
        Common families: 575
        Families in training but not in testing: 20 (2.99\%)
        Families in testing but not in training: 75 (11.19\%)
------------------------------------------------------------------
Report: 2021-08-01 00:00:00, BS: 0.5114925373134328
        Training set length: 27416, (40.92\%)
        Testing set length: 39584, (59.08\%)
        Num families in training: 596
        Num families in testing: 650
        Common families: 576
        Families in training but not in testing: 20 (2.99\%)
        Families in testing but not in training: 74 (11.04\%)
------------------------------------------------------------------
Report: 2021-09-01 00:00:00, BS: 0.7616044776119403
        Training set length: 40822, (60.93\%)
        Testing set length: 26178, (39.07\%)
        Num families in training: 654
        Num families in testing: 636
        Common families: 620
        Families in training but not in testing: 34 (5.07\%)
        Families in testing but not in training: 16 (2.39\%)
------------------------------------------------------------------
Report: 2021-10-01 00:00:00, BS: 0.8480223880597015
        Training set length: 45454, (67.84\%)
        Testing set length: 21546, (32.16\%)
        Num families in training: 661
        Num families in testing: 616
        Common families: 607
        Families in training but not in testing: 54 (8.06\%)
        Families in testing but not in training: 9 (1.34\%)
------------------------------------------------------------------
Report: 2021-11-01 00:00:00, BS: 0.8501492537313433
        Training set length: 45568, (68.01\%)
        Testing set length: 21432, (31.99\%)
        Num families in training: 661
        Num families in testing: 613
        Common families: 604
        Families in training but not in testing: 57 (8.51\%)
        Families in testing but not in training: 9 (1.34\%)
------------------------------------------------------------------
Report: 2021-12-01 00:00:00, BS: 0.8575
        Training set length: 45962, (68.6\%)
        Testing set length: 21038, (31.4\%)
        Num families in training: 661
        Num families in testing: 611
        Common families: 602
        Families in training but not in testing: 59 (8.81\%)
        Families in testing but not in training: 9 (1.34\%)
------------------------------------------------------------------
Report: 2022-01-01 00:00:00, BS: 0.7609141791044776
        Training set length: 66415, (99.13\%)
        Testing set length: 585, (0.87\%)
        Num families in training: 670
        Num families in testing: 75
        Common families: 75
        Families in training but not in testing: 595 (88.81\%)
        Families in testing but not in training: 0 (0.0\%)
------------------------------------------------------------------
Report: 2022-02-01 00:00:00, BS: 0.7600559701492539
        Training set length: 66461, (99.2\%)
        Testing set length: 539, (0.8\%)
        Num families in training: 670
        Num families in testing: 72
        Common families: 72
        Families in training but not in testing: 598 (89.25\%)
        Families in testing but not in training: 0 (0.0\%)
------------------------------------------------------------------
Report: 2022-03-01 00:00:00, BS: 0.7551492537313433
        Training set length: 66724, (99.59\%)
        Testing set length: 276, (0.41\%)
        Num families in training: 670
        Num families in testing: 56
        Common families: 56
        Families in training but not in testing: 614 (91.64\%)
        Families in testing but not in training: 0 (0.0\%)
    \end{Verbatim}

    \hypertarget{jensen-shannon-distance-and-appearing-families-considerations}{%
\subsection{Jensen-Shannon distance and \% Appearing families:
Considerations}\label{jensen-shannon-distance-and-appearing-families-considerations}}

As discussed before, in the last three bins there isn't any new
appearing family in the test set.

The drastic increase of Jensen-Shannon distance is due to the fact that
lots of families are disappearing e.g.~are seen in the training set but
not in the testing set.

In 2021-12-01 8.81\% of families are seen only on the training set,
while 88.81\% at 2022-01-01 .

Those three splits are not interesting for studying family drift.

    \begin{tcolorbox}[breakable, size=fbox, boxrule=1pt, pad at break*=1mm,colback=cellbackground, colframe=cellborder]
\prompt{In}{incolor}{9}{\boxspacing}
\begin{Verbatim}[commandchars=\\\{\}]
\PY{n}{df\PYZus{}js} \PY{o}{=} \PY{n}{pd}\PY{o}{.}\PY{n}{DataFrame}\PY{p}{(}\PY{p}{\PYZob{}}\PY{l+s+s2}{\PYZdq{}}\PY{l+s+s2}{split}\PY{l+s+s2}{\PYZdq{}}\PY{p}{:} \PY{n}{date\PYZus{}splits}\PY{p}{,} \PY{l+s+s2}{\PYZdq{}}\PY{l+s+s2}{score}\PY{l+s+s2}{\PYZdq{}}\PY{p}{:} \PY{n}{js\PYZus{}scores}\PY{p}{\PYZcb{}}\PY{p}{)}
\PY{n}{plt}\PY{o}{.}\PY{n}{figure}\PY{p}{(}\PY{n}{figsize}\PY{o}{=}\PY{p}{(}\PY{l+m+mi}{10}\PY{p}{,} \PY{l+m+mi}{7}\PY{p}{)}\PY{p}{)}
\PY{n}{ax} \PY{o}{=} \PY{n}{sns}\PY{o}{.}\PY{n}{barplot}\PY{p}{(}\PY{n}{data}\PY{o}{=}\PY{n}{df\PYZus{}js}\PY{p}{,} \PY{n}{x}\PY{o}{=}\PY{l+s+s2}{\PYZdq{}}\PY{l+s+s2}{split}\PY{l+s+s2}{\PYZdq{}}\PY{p}{,} \PY{n}{y}\PY{o}{=}\PY{l+s+s2}{\PYZdq{}}\PY{l+s+s2}{score}\PY{l+s+s2}{\PYZdq{}}\PY{p}{)}
\PY{n}{plt}\PY{o}{.}\PY{n}{title}\PY{p}{(}\PY{l+s+s2}{\PYZdq{}}\PY{l+s+s2}{Jensen\PYZhy{}Shannon distance between train/test data}\PY{l+s+s2}{\PYZdq{}}\PY{p}{)}
\PY{n}{plt}\PY{o}{.}\PY{n}{xticks}\PY{p}{(}\PY{p}{[}\PY{p}{]}\PY{p}{)}
\PY{n}{plt}\PY{o}{.}\PY{n}{show}\PY{p}{(}\PY{p}{)}
\end{Verbatim}
\end{tcolorbox}

    \begin{center}
    \adjustimage{max size={0.9\linewidth}{0.9\paperheight}}{best_split_files/best_split_14_0.png}
    \end{center}
    { \hspace*{\fill} \\}
    
    \begin{tcolorbox}[breakable, size=fbox, boxrule=1pt, pad at break*=1mm,colback=cellbackground, colframe=cellborder]
\prompt{In}{incolor}{10}{\boxspacing}
\begin{Verbatim}[commandchars=\\\{\}]
\PY{n}{plt}\PY{o}{.}\PY{n}{figure}\PY{p}{(}\PY{n}{figsize}\PY{o}{=}\PY{p}{(}\PY{l+m+mi}{10}\PY{p}{,} \PY{l+m+mi}{7}\PY{p}{)}\PY{p}{)}
\PY{n}{df\PYZus{}af} \PY{o}{=} \PY{n}{pd}\PY{o}{.}\PY{n}{DataFrame}\PY{p}{(}\PY{p}{\PYZob{}}\PY{l+s+s2}{\PYZdq{}}\PY{l+s+s2}{split}\PY{l+s+s2}{\PYZdq{}}\PY{p}{:} \PY{n}{date\PYZus{}splits}\PY{p}{,} \PY{l+s+s2}{\PYZdq{}}\PY{l+s+s2}{score}\PY{l+s+s2}{\PYZdq{}}\PY{p}{:} \PY{n}{perc\PYZus{}app\PYZus{}families}\PY{p}{\PYZcb{}}\PY{p}{)}
\PY{n}{ax} \PY{o}{=} \PY{n}{sns}\PY{o}{.}\PY{n}{barplot}\PY{p}{(}\PY{n}{data}\PY{o}{=}\PY{n}{df\PYZus{}af}\PY{p}{,} \PY{n}{x}\PY{o}{=}\PY{l+s+s2}{\PYZdq{}}\PY{l+s+s2}{split}\PY{l+s+s2}{\PYZdq{}}\PY{p}{,} \PY{n}{y}\PY{o}{=}\PY{l+s+s2}{\PYZdq{}}\PY{l+s+s2}{score}\PY{l+s+s2}{\PYZdq{}}\PY{p}{)}
\PY{n}{plt}\PY{o}{.}\PY{n}{title}\PY{p}{(}\PY{l+s+s2}{\PYZdq{}}\PY{l+s+s2}{\PYZpc{}}\PY{l+s+s2}{ Appearing families in testing set}\PY{l+s+s2}{\PYZdq{}}\PY{p}{)}
\PY{n}{plt}\PY{o}{.}\PY{n}{xticks}\PY{p}{(}\PY{p}{[}\PY{p}{]}\PY{p}{)}
\PY{n}{plt}\PY{o}{.}\PY{n}{show}\PY{p}{(}\PY{p}{)}
\end{Verbatim}
\end{tcolorbox}

    \begin{center}
    \adjustimage{max size={0.9\linewidth}{0.9\paperheight}}{best_split_files/best_split_15_0.png}
    \end{center}
    { \hspace*{\fill} \\}
    
    \begin{tcolorbox}[breakable, size=fbox, boxrule=1pt, pad at break*=1mm,colback=cellbackground, colframe=cellborder]
\prompt{In}{incolor}{11}{\boxspacing}
\begin{Verbatim}[commandchars=\\\{\}]
\PY{n}{df\PYZus{}bs}\PY{p}{[}\PY{l+s+s2}{\PYZdq{}}\PY{l+s+s2}{label}\PY{l+s+s2}{\PYZdq{}}\PY{p}{]} \PY{o}{=} \PY{l+s+s2}{\PYZdq{}}\PY{l+s+s2}{80\PYZhy{}20 Balance score}\PY{l+s+s2}{\PYZdq{}}
\PY{n}{df\PYZus{}js}\PY{p}{[}\PY{l+s+s2}{\PYZdq{}}\PY{l+s+s2}{label}\PY{l+s+s2}{\PYZdq{}}\PY{p}{]} \PY{o}{=} \PY{l+s+s2}{\PYZdq{}}\PY{l+s+s2}{Jensen Shannon distance}\PY{l+s+s2}{\PYZdq{}}
\PY{n}{df\PYZus{}af}\PY{p}{[}\PY{l+s+s2}{\PYZdq{}}\PY{l+s+s2}{label}\PY{l+s+s2}{\PYZdq{}}\PY{p}{]} \PY{o}{=} \PY{l+s+s2}{\PYZdq{}}\PY{l+s+s2}{\PYZpc{}}\PY{l+s+s2}{ Appearing families}\PY{l+s+s2}{\PYZdq{}}

\PY{n}{df\PYZus{}concat\PYZus{}scores} \PY{o}{=} \PY{n}{pd}\PY{o}{.}\PY{n}{concat}\PY{p}{(}\PY{p}{[}\PY{n}{df\PYZus{}bs}\PY{p}{,} \PY{n}{df\PYZus{}af}\PY{p}{]}\PY{p}{)}
\PY{n}{df\PYZus{}concat\PYZus{}scores} \PY{o}{=} \PY{n}{df\PYZus{}concat\PYZus{}scores}\PY{p}{[}\PY{n}{df\PYZus{}concat\PYZus{}scores}\PY{p}{[}\PY{l+s+s2}{\PYZdq{}}\PY{l+s+s2}{split}\PY{l+s+s2}{\PYZdq{}}\PY{p}{]} \PY{o}{\PYZgt{}} \PY{n}{t\PYZus{}focus}\PY{p}{]}

\PY{n}{plt}\PY{o}{.}\PY{n}{figure}\PY{p}{(}\PY{n}{figsize}\PY{o}{=}\PY{p}{(}\PY{l+m+mi}{10}\PY{p}{,} \PY{l+m+mi}{7}\PY{p}{)}\PY{p}{)}
\PY{n}{ax} \PY{o}{=} \PY{n}{sns}\PY{o}{.}\PY{n}{barplot}\PY{p}{(}\PY{n}{data}\PY{o}{=}\PY{n}{df\PYZus{}concat\PYZus{}scores}\PY{p}{,} \PY{n}{x}\PY{o}{=}\PY{l+s+s2}{\PYZdq{}}\PY{l+s+s2}{split}\PY{l+s+s2}{\PYZdq{}}\PY{p}{,} \PY{n}{hue}\PY{o}{=}\PY{l+s+s2}{\PYZdq{}}\PY{l+s+s2}{label}\PY{l+s+s2}{\PYZdq{}}\PY{p}{,} \PY{n}{y}\PY{o}{=}\PY{l+s+s2}{\PYZdq{}}\PY{l+s+s2}{score}\PY{l+s+s2}{\PYZdq{}}\PY{p}{)}
\PY{n}{plt}\PY{o}{.}\PY{n}{title}\PY{p}{(}\PY{l+s+s2}{\PYZdq{}}\PY{l+s+s2}{80\PYZhy{}20 Balance Score and }\PY{l+s+s2}{\PYZpc{}}\PY{l+s+s2}{ Appearing families: a comparison}\PY{l+s+s2}{\PYZdq{}}\PY{p}{)}
\PY{n}{plt}\PY{o}{.}\PY{n}{xticks}\PY{p}{(}\PY{n}{rotation}\PY{o}{=}\PY{l+m+mi}{90}\PY{p}{)}
\PY{n}{plt}\PY{o}{.}\PY{n}{show}\PY{p}{(}\PY{p}{)}
\end{Verbatim}
\end{tcolorbox}

    \begin{center}
    \adjustimage{max size={0.9\linewidth}{0.9\paperheight}}{best_split_files/best_split_16_0.png}
    \end{center}
    { \hspace*{\fill} \\}
    
    \hypertarget{best-split-based-on-80-20-training-test-balance-score-and-appearing-families}{%
\subsection{Best split based on 80-20 Training-Test Balance Score and \%
Appearing
families}\label{best-split-based-on-80-20-training-test-balance-score-and-appearing-families}}

Using \(BS > 0.5\) e.g.~40.18\% of the training set length as a basis,
the best split is computed using \(BS\) + \% Appearing families as
optimization criteria.

    \begin{tcolorbox}[breakable, size=fbox, boxrule=1pt, pad at break*=1mm,colback=cellbackground, colframe=cellborder]
\prompt{In}{incolor}{12}{\boxspacing}
\begin{Verbatim}[commandchars=\\\{\}]
\PY{n}{df\PYZus{}merge\PYZus{}scores} \PY{o}{=} \PY{n}{pd}\PY{o}{.}\PY{n}{merge}\PY{p}{(}\PY{n}{left}\PY{o}{=}\PY{n}{df\PYZus{}bs}\PY{p}{,} \PY{n}{right}\PY{o}{=}\PY{n}{df\PYZus{}af}\PY{p}{,} \PY{n}{on}\PY{o}{=}\PY{l+s+s2}{\PYZdq{}}\PY{l+s+s2}{split}\PY{l+s+s2}{\PYZdq{}}\PY{p}{)}
\PY{n}{df\PYZus{}merge\PYZus{}scores} \PY{o}{=} \PY{n}{df\PYZus{}merge\PYZus{}scores}\PY{p}{[}\PY{n}{df\PYZus{}merge\PYZus{}scores}\PY{p}{[}\PY{l+s+s2}{\PYZdq{}}\PY{l+s+s2}{split}\PY{l+s+s2}{\PYZdq{}}\PY{p}{]} \PY{o}{\PYZgt{}} \PY{n}{t\PYZus{}focus}\PY{p}{]}
\PY{n}{s\PYZus{}scores} \PY{o}{=} \PY{n}{df\PYZus{}merge\PYZus{}scores}\PY{p}{[}\PY{l+s+s2}{\PYZdq{}}\PY{l+s+s2}{score\PYZus{}x}\PY{l+s+s2}{\PYZdq{}}\PY{p}{]} \PY{o}{+} \PY{n}{df\PYZus{}merge\PYZus{}scores}\PY{p}{[}\PY{l+s+s2}{\PYZdq{}}\PY{l+s+s2}{score\PYZus{}y}\PY{l+s+s2}{\PYZdq{}}\PY{p}{]}

\PY{n}{max\PYZus{}v}\PY{p}{,} \PY{n}{max\PYZus{}v\PYZus{}idx} \PY{o}{=} \PY{n}{np}\PY{o}{.}\PY{n}{max}\PY{p}{(}\PY{n}{s\PYZus{}scores}\PY{p}{)}\PY{p}{,} \PY{n}{np}\PY{o}{.}\PY{n}{argmax}\PY{p}{(}\PY{n}{s\PYZus{}scores}\PY{p}{)}
\PY{n}{best\PYZus{}split} \PY{o}{=} \PY{n+nb}{list}\PY{p}{(}\PY{n}{df\PYZus{}merge\PYZus{}scores}\PY{p}{[}\PY{l+s+s1}{\PYZsq{}}\PY{l+s+s1}{split}\PY{l+s+s1}{\PYZsq{}}\PY{p}{]}\PY{p}{)}\PY{p}{[}\PY{n}{max\PYZus{}v\PYZus{}idx}\PY{p}{]}
\PY{n}{print\PYZus{}statistics}\PY{p}{(}\PY{n}{df\PYZus{}scores}\PY{p}{,} \PY{n}{best\PYZus{}split}\PY{p}{,} \PY{n}{best\PYZus{}split}\PY{p}{)}
\end{Verbatim}
\end{tcolorbox}

    \begin{Verbatim}[commandchars=\\\{\}]
------------------------------------------------------------------
Report: 2021-12-01 00:00:00
        Training set length: 45962, (68.6\%)
        Testing set length: 21038, (31.4\%)
        Num families in training: 661
        Num families in testing: 611
        Common families: 602
        Families in training but not in testing: 59 (8.81\%)
        Families in testing but not in training: 9 (1.34\%)
    \end{Verbatim}

    \hypertarget{conclusions}{%
\subsection{Conclusions}\label{conclusions}}

Using a Training-Testing Balance score that increases when the training
set length is approaching 80\% of the overall data + \% of appearing
families in the test set gives 2021-12-01 as the best split, with new 9
families seen in the testing set.

As described before, there's another time split that is interesting to
consider: 2021-09-01. The training set here is composed of
\textasciitilde60\% of data points (still has a large percentage of
samples) and 16 families are appearing in the testing set, as opposed to
9.

This indicates that the Training-Testing balance score isn't much
appropriate because it linearly decrease as the percentage deviates from
80\%, impacting too much negatively for still quite good splits.

Furthermore, equidistant splits of one day might be considered on later
analysis, especially for 2021-08, where a sudden increase of training
set length verifies (40.92\% to 60.93\%).


    % Add a bibliography block to the postdoc
    
    
    
\end{document}
